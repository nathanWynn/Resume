% Cover letter using letter.sty
\documentclass{letter} % Uses 10pt
%Use \documentstyle[newcent]{letter} for New Century Schoolbook postscript font
% the following commands control the margins:
\topmargin=-1in    % Make letterhead start about 1 inch from top of page 
\textheight=8in  % text height can be bigger for a longer letter
\oddsidemargin=0pt % leftmargin is 1 inch
\textwidth=6.5in   % textwidth of 6.5in leaves 1 inch for right margin

\begin{document}

\signature{Nathan W Wynn}           % name for signature 
\longindentation=0pt                       % needed to get closing flush left
\let\raggedleft\raggedright                % needed to get date flush left
 
 
\begin{letter}%{Ms. Terri Roberts \\
%Senior Staff Recruiter \\
%XYZ Corporation \\
%Rt. 56 \\
%Anytown, New Jersey 05867}


\begin{flushleft}
{\large\bf Nathan W Wynn}
\end{flushleft}
\medskip\hrule height 1pt
\begin{flushright}
\hfill 1256 Bethlehem Rd, Statham, GA 30666 \\
\hfill (762) 436-9384
\end{flushright} 
\vfill % forces letterhead to top of page

%"Describe a problem in computing that you encountered and how you solved it. Your statement may address any of the following: why the problem was interesting, what your specific contribution to the solution was, what challenges you encountered, and what skills (technical or otherwise) you learned."%
 %
 
\noindent One of the projects I am currently participating in is a Virtual Reality Lunar Rover Simulator research project with four of my peers. The purpose of the project is to simulate an accurate, networked experience to train users how to operate an unmanned lunar rover prototype in a virtual environment rather than designing and building a physical rover. This problem is particularly interesting, since we were given the freedom to design the rover to suit our specific needs in performance data gathering.

\noindent The problem I encountered and solved was the locomotion system used to control the rover. There are quite a few locomotion paradigms commonly used in virtual reality, such as teleportation, flight, and real walking, however, none of the existing locomotion schemes that we tested fit our specific needs. The specific problems we needed to solve involving locomotion were avoiding motion sickness, facilitating the ability to interact with the environment, and providing precision in navigating the rough terrain. I proposed a scheme such that the body of the rover and the head (where the camera and user's view were located) would be treated as two separate entities. Unlike common driving paradigms, the head's transform in world space would be tied to that of the body, but the head's rotation would be entirely dependent on the user's head mounted display. This solved the problem of creating motion sickness by only rotating the head in the virtual environment with the physical consent of the user. Next, my solution for navigation was to utilize both joysticks as "thrusters" of sorts. For example, tilting the left joystick forward or backward would activate the wheels on the left hand side of the rover and move the tracks in the corresponding direction. This allowed for precise in-place rotation and maneuvering as well as giving the user the ability to both focus on a particular object and maneuver simultaneously. Further, this allowed us to use the tracking of the controllers to lift and interact with objects in the virtual environment just like a user would in the real world.

\noindent One of the challenges associated with this locomotion paradigm was the design aspect. We needed a locomotion system that had parallels to existing technology, but at the same time provided new ways of interacting with the world around it. Specifically, I was challenged in creating a seamless relationship between interaction and locomotion, as these two elements were critical to the project. My solution provided a more optimal relationship between the two than any other common locomotion paradigm we encountered.

\noindent Through working on this project, I had the opportunity to further my collaboration skills in implementing the simulation with all of my peers. I also learned how to use existing solutions to a problem as a baseline for creating new specialized solutions that better suited our needs. Further, I strengthened my programming skills by implementing the controls for this system using Unity and C$\#$.



 
%\encl{}  				% Enclosures

\end{letter}
 

\end{document}






